\documentclass{article}

\usepackage{arxiv}

\usepackage[utf8]{inputenc} % allow utf-8 input
\usepackage[T1]{fontenc}    % use 8-bit T1 fonts
\usepackage{lmodern}        % https://github.com/rstudio/rticles/issues/343
\usepackage{hyperref}       % hyperlinks
\usepackage{url}            % simple URL typesetting
\usepackage{booktabs}       % professional-quality tables
\usepackage{amsfonts}       % blackboard math symbols
\usepackage{nicefrac}       % compact symbols for 1/2, etc.
\usepackage{microtype}      % microtypography
\usepackage{graphicx}

\title{Teaching statistical inference in the age of financial technology}

\author{
    Barry Quinn
    \thanks{A special thanks to Dr Alan Hanna for his insightful comments.}
   \\
    Queens Management School \\
    Queen's University Belfast \\
  Belfast \\
  \texttt{\href{mailto:b.quinn@qub.ac.uk}{\nolinkurl{b.quinn@qub.ac.uk}}} \\
  }



% Pandoc citation processing

\usepackage{booktabs} \usepackage{longtable} \usepackage{array} \usepackage{multirow} \usepackage[table]{xcolor} \usepackage{wrapfig} \usepackage{float} \floatplacement{figure}{H}


\begin{document}
\maketitle

\def\tightlist{}


\begin{abstract}
This paper seeks to understand the challenges of teaching statistical
inference in finance in the computer age. We argue that the unstoppable
algorithmic transformation of financial services and the developing
field of machine learning provide an opportunity to reboot financial
econometrics for the financial technology era. We argue it is time for a
rethink how we can extract reliable statistical inference from financial
data given the proliferation of computing, Big financial data and the
unstoppable algorithmisation of the finance industry. Firstly, we
agnostically profiling the modelling paradigms avaliable in the FinTech
era. Next, we establish the developments in statistical inference in the
digital age. Finally, we consider placing computation as a central
tenant in finance curricula and discuss the infrastructure and tools
involved. We illustrate a use case where the infrastructure is
on-boarded in a cloud computing suite with enterprise-level server
software. We are not arguing that finance is computation; instead, by
placing computation as a frictionless part of the curriculum, students
can engage with the full suite of state-of-the-art inferential tools
available to financial data science practitioners.
\end{abstract}

\keywords{
    Finance education
   \and
    Financial technology
   \and
    Statistical inference
   \and
    Financial data science
   \and
    Financial machine learning
   \and
    Econometrics
   \and
    Cloud computing
   \and
    Employability
  }



\bibliographystyle{unsrt}
\bibliography{}


\end{document}
